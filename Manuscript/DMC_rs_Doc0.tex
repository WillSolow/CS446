% Modified Skye Rhomberg 9/30/20


%%%%%%%%%%%%%%%%%%%%%%%%%%%%%%%%%%%%%%%%%%%%%%%%%%%%%%%%%%%%%%%%%%%%%
%% This is a (brief) model paper using the achemso class
%% The document class accepts keyval options, which should include
%% the target journal and optionally the manuscript type.
%%%%%%%%%%%%%%%%%%%%%%%%%%%%%%%%%%%%%%%%%%%%%%%%%%%%%%%%%%%%%%%%%%%%%
\documentclass[journal=jacsat,manuscript=article]{achemso}

%%%%%%%%%%%%%%%%%%%%%%%%%%%%%%%%%%%%%%%%%%%%%%%%%%%%%%%%%%%%%%%%%%%%%
%% Place any additional packages needed here.  Only include packages
%% which are essential, to avoid problems later.
%%%%%%%%%%%%%%%%%%%%%%%%%%%%%%%%%%%%%%%%%%%%%%%%%%%%%%%%%%%%%%%%%%%%%
\usepackage{chemformula} % Formula subscripts using \ch{}
\usepackage[T1]{fontenc} % Use modern font encodings

%%%%%%%%%%%%%%%%%%%%%%%%%%%%%%%%%%%%%%%%%%%%%%%%%%%%%%%%%%%%%%%%%%%%%
%% If issues arise when submitting your manuscript, you may want to
%% un-comment the next line.  This provides information on the
%% version of every file you have used.
%%%%%%%%%%%%%%%%%%%%%%%%%%%%%%%%%%%%%%%%%%%%%%%%%%%%%%%%%%%%%%%%%%%%%
%%\listfiles

%%%%%%%%%%%%%%%%%%%%%%%%%%%%%%%%%%%%%%%%%%%%%%%%%%%%%%%%%%%%%%%%%%%%%
%% Place any additional macros here.  Please use \newcommand* where
%% possible, and avoid layout-changing macros (which are not used
%% when typesetting).
%%%%%%%%%%%%%%%%%%%%%%%%%%%%%%%%%%%%%%%%%%%%%%%%%%%%%%%%%%%%%%%%%%%%%
\newcommand*\mycommand[1]{\texttt{\emph{#1}}}

%%%%%%%%%%%%%%%%%%%%%%%%%%%%%%%%%%%%%%%%%%%%%%%%%%%%%%%%%%%%%%%%%%%%%
%% Meta-data block
%% ---------------
%% Each author should be given as a separate \author command.
%%
%% Corresponding authors should have an e-mail given after the author
%% name as an \email command. Phone and fax numbers can be given
%% using \phone and \fax, respectively; this information is optional.
%%
%% The affiliation of authors is given after the authors; each
%% \affiliation command applies to all preceding authors not already
%% assigned an affiliation.
%%
%% The affiliation takes an option argument for the short name.  This
%% will typically be something like "University of Somewhere".
%%
%% The \altaffiliation macro should be used for new address, etc.
%% On the other hand, \alsoaffiliation is used on a per author basis
%% when authors are associated with multiple institutions.
%%%%%%%%%%%%%%%%%%%%%%%%%%%%%%%%%%%%%%%%%%%%%%%%%%%%%%%%%%%%%%%%%%%%%
\author{Skye Rhomberg}
\author{Will Solow}
\author{Eric Aaron}
\email{sorhom22@colby.edu}
\phone{+123 (0)123 4445556}
\affiliation[Colby College]
{Department of Computer Science, Colby College, Waterville, ME}
%%%%%%%%%%%%%%%%%%%%%%%%%%%%%%%%%%%%%%%%%%%%%%%%%%%%%%%%%%%%%%%%%%%%%
%% The document title should be given as usual. Some journals require
%% a running title from the author: this should be supplied as an
%% optional argument to \title.
%%%%%%%%%%%%%%%%%%%%%%%%%%%%%%%%%%%%%%%%%%%%%%%%%%%%%%%%%%%%%%%%%%%%%
\title[An \textsf{achemso} demo]
  {Small Assignment 10/30 - Code Validation}

%%%%%%%%%%%%%%%%%%%%%%%%%%%%%%%%%%%%%%%%%%%%%%%%%%%%%%%%%%%%%%%%%%%%%
%% Some journals require a list of abbreviations or keywords to be
%% supplied. These should be set up here, and will be printed after
%% the title and author information, if needed.
%%%%%%%%%%%%%%%%%%%%%%%%%%%%%%%%%%%%%%%%%%%%%%%%%%%%%%%%%%%%%%%%%%%%%
\abbreviations{IR,NMR,UV}
\keywords{American Chemical Society, \LaTeX}

%%%%%%%%%%%%%%%%%%%%%%%%%%%%%%%%%%%%%%%%%%%%%%%%%%%%%%%%%%%%%%%%%%%%%
%% The manuscript does not need to include \maketitle, which is
%% executed automatically.
%%%%%%%%%%%%%%%%%%%%%%%%%%%%%%%%%%%%%%%%%%%%%%%%%%%%%%%%%%%%%%%%%%%%%
\begin{document}

%%%%%%%%%%%%%%%%%%%%%%%%%%%%%%%%%%%%%%%%%%%%%%%%%%%%%%%%%%%%%%%%%%%%%
%% The "tocentry" environment can be used to create an entry for the
%% graphical table of contents. It is given here as some journals
%% require that it is printed as part of the abstract page. It will
%% be automatically moved as appropriate.
%%%%%%%%%%%%%%%%%%%%%%%%%%%%%%%%%%%%%%%%%%%%%%%%%%%%%%%%%%%%%%%%%%%%%
%\begin{tocentry}
%
%Some journals require a graphical entry for the Table of Contents.
%This should be laid out ``print ready'' so that the sizing of the
%text is correct.
%
%Inside the \texttt{tocentry} environment, the font used is Helvetica
%8\,pt, as required by \emph{Journal of the American Chemical
%Society}.
%
%The surrounding frame is 9\,cm by 3.5\,cm, which is the maximum
%permitted for  \emph{Journal of the American Chemical Society}
%graphical table of content entries. The box will not resize if the
%content is too big: instead it will overflow the edge of the box.
%
%This box and the associated title will always be printed on a
%separate page at the end of the document.
%
%\end{tocentry}

%%%%%%%%%%%%%%%%%%%%%%%%%%%%%%%%%%%%%%%%%%%%%%%%%%%%%%%%%%%%%%%%%%%%%
%% The abstract environment will automatically gobble the contents
%% if an abstract is not used by the target journal.
%%%%%%%%%%%%%%%%%%%%%%%%%%%%%%%%%%%%%%%%%%%%%%%%%%%%%%%%%%%%%%%%%%%%%
\begin{abstract}
We~\cite{Anderson1975} give an overview of the Diffusion Monte Carlo (DMC) algorithm and its applications into finding solutions to the Schrodinger Equation when no analytical solution is available. An implementation of the DMC algorithm is presented with a specific focus on understanding the behaviour of Clathrate Hydrates, a complex, multi-atomic structure often made up of water molecules and a greenhouse gas. Given the many ways of representing such a system, we present conditions for which different representations produce the computationally efficient results. We show how Importance Sampling can be used to reduce some of the computational inefficiency that abounds when working in high dimensional data sets with results that mirror those of a straightforward DMC implementation. Finally, we provide an analysis of the above algorithm when implemented in a Object Oriented versus script-based format to conclude that script-based programming is more computationally efficient in this scope.
\end{abstract}

%%%%%%%%%%%%%%%%%%%%%%%%%%%%%%%%%%%%%%%%%%%%%%%%%%%%%%%%%%%%%%%%%%%%%
%% Start the main part of the manuscript here.
%%%%%%%%%%%%%%%%%%%%%%%%%%%%%%%%%%%%%%%%%%%%%%%%%%%%%%%%%%%%%%%%%%%%%
\section{Introduction}
We introduce an efficient, generalizable, and validated implementation of the Diffusion Monte Carlo (DMC) algorithm utilizing NumPy. We discuss the problems faced in the world of Computational Sciences in writing validated, efficient, and widely usable code and how our implementation of the DMC algorithm furthers the work in these areas. On the topic of efficiency, we compare an Object-Oriented approach with a scripting approach, and discuss the pros and cons of each implementation in the context of verifiability, efficiency and generalizability. 

The Shrodinger equation is near-impossible to solve for complicated molecular systems. To date, the DMC algorithm is one of the best proposed algorithms for creating a tight approximation of the Schrodinger equation for complex systems. Given the complexity of the DMC algorithm and its potential for time growth in large systems, studying faster implementations of it is an important area of research. When it comes to computational studies, time barriers often prevent work from being done as the computer is required to analyze a large amount of possible simulations.

However, a fast implementation is not valuable unless it is correct. We provide techniques for how we validated our work and hope that these techniques and practices can be a staple of all work done in the Computational Science field. 
\section{Related Work}

This semester, we code a python based implementation of the Diffusion Monte Carlo algorithm originally presented in Anderson 1975. Our implementation of the DMC algorithm can be directly attributed to Anderson's work, as he was one of the first to propose such a method. At the time, other methods were seen as more computationally efficient for a given level of accuracy. We take Anderson's algorithm and refine it in the scope of being efficient for computing the ground level energy and bond strength of Clathrate Hydrates. 
While Anderson was initially only concerned with the calculation of the ground energy of H3, his simplifying approach to the 1D and 6D systems greatly influenced our approach in initial versions of our implementation. We improve on his work by designing a more efficient, vector-based implementation using the numpy library provided by python. We hope to extend this approach to the task of modelling more complicated Clathrate Hydrates which require an increased computational load.
As Anderson points out, the Schrodinger Equation is only analytically solvable for certain, small systems. His DMC algorithm provides such a way to approximate solutions to complex systems. The best solution to modelling Clathrate Hydrates, given the lack of an analytical solution, is an open question. We hope to build off of Anderson's work to provide a computationally efficient method for the solution of this problem.


\section{Modeling}
\subsection{The Model}
The DMC model is not novel. It~\cite{Anderson1975} was first introduced over 40 years ago and has been refined since. At a high level, the model takes an array of walkers, each representing a possible state of the system, and propagates the atoms in each walker. After propagation, the potential energy of each system is calculated. Systems with potential energy that is too high are deleted, while systems with reasonable potential energy are replicated. Given the pseudo-randomness leveraged in code-based implementations of the DMC algorithm, this serves to approximate every possible configuration of the system, within the bounds that it would normally exist in nature.

Over time, the rolling average the potential energy of all systems that are valid converge to the zero point energy, or the ground state energy of the system. This produces an approximate solution of the Schrodinger equation, which is what the DMC algorithm is modelling. 
\subsection{Implementation}
In our implementation of DMC, we write a script-based implementation in Python, relying heavily on the NumPy library, which provides vectorization for fast operations on large matrices. The most important variable in the code is a 4D array which stores the positions of each atom in each molecule in each walker. As we move through the simulation loop of the algorithm, we operate on the walker array to delete and replicate walkers as required by the algorithm.

Notably, the function that calculates potential energy is drastically different from molecular system to molecular system, so the main simulation loop uses calls to the potential energy fucntion for the purpose of extendability. From system to system, the only other varying part of our code is how we propagate each walker. This could potentially change with each system, although we hope to find a data structure that allows our code to remain effiicient without needing to change how the walkers are propagated at each step. Otherwise, the entire model is controlled by the initial variables of the simulation.

Through the simulation loop, rely heavily on the broadcasting features of NumPy to different array dimensions to figure out which walkers are to be deleted or replicated. We like to think that this is done quite cleanly, and do it in a mere 11 lines of code after the walkers are propagated. This efficiency is appealing and has been why we are hesitant to walkk away from the 4D array based implementation that we provide. 
\subsection{Code Validation}

In any implementation of an algorithm, rigorous testing is required to ensure that the code is producing accurate results. Ideally, data generated from a simulation would be corroborated with data collected experimentally. However, due to the nature of the study of particles, outside of the simplest systems, it is difficult to accurately calculate the ground state energy. As such, results generated from the DMC algorithm cannot be verified against real world data. 

Instead, we turn to manual validation of the code. An outline of the process is as follows: a set of walkers is generated. The validator calculates the potential energy and the reference energy. Using a randomly generated set of thresholds, the validator determines which walkers should be deleted and replicated, and then compares the final array to the final array produced by the algorithm. To be sure of correctness, this process was repeated multiple times with sets of ten walkers over five time steps. These groups of calculations should be comprehensive enough to validate the correctness of the algorithm due to the stochastic nature of the simulation, and generalize well to larger groups of walkers. 

As a final measure, the implementaion of the DMC algorithm was tested to model the carbon monoxide bond, given that the Zero-Point energy of the CO bond is well known. When tested over multiple simulations, the 1,000 step rolling average varied on the order of  $1\cdot10^{-4}$, with a variance on the order of $1\cdot10^{-3}$. Based on prior DMC implementations, this amount of inaccuracy is normal.  

\section{Simulations}

\section{Discussion}
When creating a piece of code in the field of Computational Science, its worth is often based on how extendable it is to other projects of interest. In the case of our work, we provide a fast, script based implementation of the DMC algorithm by leveraging the vectorization provided in the NumPy library. Currently, our work relies on the assumption that the molecules in the modeled system are homogeneous. Clearly, this is not a particularly extendable piece of code, as there are many other interesting systems made up of hetereogeneous molecules. 

This begs us to consider how to create generalizable code that can be used in a variety of circumstances without sacrificing efficiency. Observe that NumPy works well when dealing with arrays of many dimensions. As soon as we change the sizes of the molecules in a walker, the dimensions of our array are no longer even, and so NumPy cannot cleanly vectorize the solution in the 4D data structure that we have presented up to this point. From a programming point of view, it would not be particularly difficult to represent each walker on a singular array dimesion. However, this also reduces the extendablility of our code as it makes the potential energy function extraordinarily difficult to calculate, given that each molecule and atom corresponds to a particular index on the same array dimesion. 

We hope that the users of our code are domain experts, and want to create a scientific instrument that is easy and intuitive for them to use. As such, we continue to try and refine our ideas to create an implementation of DMC that balances both efficiency and ease of use. 

\section{Conclusion}
% Everything After this comes later
%\section{Results and discussion}
%
%\subsection{Outline}
%
%The document layout should follow the style of the journal concerned.
%Where appropriate, sections and subsections should be added in the
%normal way. If the class options are set correctly, warnings will be
%given if these should not be present.
%
%\subsection{References}
%
%The class makes various changes to the way that references are
%handled.  The class loads \textsf{natbib}, and also the
%appropriate bibliography style.  References can be made using
%the normal method; the citation should be placed before any
%punctuation, as the class will move it if using a superscript
%citation style
%\cite{Mena2000,Abernethy2003,Friedman-Hill2003,EuropeanCommission2008}.
%The use of \textsf{natbib} allows the use of the various citation
%commands of that package: \citeauthor{Abernethy2003} have shown
%something, in \citeyear{Cotton1999}, or as given by
%Ref.~\citenum{Mena2000}.  Long lists of authors will be
%automatically truncated in most article formats, but not in
%supplementary information or reviews \cite{Pople2003}. If you
%encounter problems with the citation macros, please check that
%your copy of \textsf{natbib} is up to date. The demonstration
%database file \texttt{achemso-demo.bib} shows how to complete
%entries correctly. Notice that ``\latin{et al.}'' is auto-formatted
%using the \texttt{\textbackslash latin} command.
%
%Multiple citations to be combined into a list can be given as
%a single citation.  This uses the \textsf{mciteplus} package
%\cite{Johnson1972,*Arduengo1992,*Eisenstein2005,*Arduengo1994}.
%Citations other than the first of the list should be indicated
%with a star. If the \textsf{mciteplus} package is not installed,
%the standard bibliography tools will still work but starred
%references will be ignored. Individual references can be referred
%to using \texttt{\textbackslash mciteSubRef}:
%``ref.~\mciteSubRef{Eisenstein2005}''.
%
%The class also handles notes to be added to the bibliography.  These
%should be given in place in the document \bibnote{This is a note.
%The text will be moved the the references section.  The title of the
%section will change to ``Notes and References''.}.  As with
%citations, the text should be placed before punctuation.  A note is
%also generated if a citation has an optional note.  This assumes that
%the whole work has already been cited: odd numbering will result if
%this is not the case \cite[p.~1]{Cotton1999}.
%
%\subsection{Floats}
%
%New float types are automatically set up by the class file.  The
%means graphics are included as follows (Scheme~\ref{sch:example}).  As
%illustrated, the float is ``here'' if possible.
%\begin{scheme}
%  Your scheme graphic would go here: \texttt{.eps} format\\
%  for \LaTeX\, or \texttt{.pdf} (or \texttt{.png}) for pdf\LaTeX\\
%  \textsc{ChemDraw} files are best saved as \texttt{.eps} files:\\
%  these can be scaled without loss of quality, and can be\\
%  converted to \texttt{.pdf} files easily using \texttt{eps2pdf}.\\
%  %\includegraphics{graphic}
%  \caption{An example scheme}
%  \label{sch:example}
%\end{scheme}
%
%\begin{figure}
%  As well as the standard float types \texttt{table}\\
%  and \texttt{figure}, the class also recognises\\
%  \texttt{scheme}, \texttt{chart} and \texttt{graph}.
%  \caption{An example figure}
%  \label{fgr:example}
%\end{figure}
%
%Charts, figures and schemes do not necessarily have to be labelled or
%captioned.  However, tables should always have a title. It is
%possible to include a number and label for a graphic without any
%title, using an empty argument to the \texttt{\textbackslash caption}
%macro.
%
%The use of the different floating environments is not required, but
%it is intended to make document preparation easier for authors. In
%general, you should place your graphics where they make logical
%sense; the production process will move them if needed.
%
%\subsection{Math(s)}
%
%The \textsf{achemso} class does not load any particular additional
%support for mathematics.  If packages such as \textsf{amsmath} are
%required, they should be loaded in the preamble.  However,
%the basic \LaTeX\ math(s) input should work correctly without
%this.  Some inline material \( y = mx + c \) or $ 1 + 1 = 2 $
%followed by some display. \[ A = \pi r^2 \]
%
%It is possible to label equations in the usual way (Eq.~\ref{eqn:example}).
%\begin{equation}
%  \frac{\mathrm{d}}{\mathrm{d}x} \, r^2 = 2r \label{eqn:example}
%\end{equation}
%This can also be used to have equations containing graphical
%content. To align the equation number with the middle of the graphic,
%rather than the bottom, a minipage may be used.
%\begin{equation}
%  \begin{minipage}[c]{0.80\linewidth}
%    \centering
%    As illustrated here, the width of \\
%    the minipage needs to allow some  \\
%    space for the number to fit in to.
%    %\includegraphics{graphic}
%  \end{minipage}
%  \label{eqn:graphic}
%\end{equation}
%
%\section{Experimental}
%
%The usual experimental details should appear here.  This could
%include a table, which can be referenced as Table~\ref{tbl:example}.
%Notice that the caption is positioned at the top of the table.
%\begin{table}
%  \caption{An example table}
%  \label{tbl:example}
%  \begin{tabular}{ll}
%    \hline
%    Header one  & Header two  \\
%    \hline
%    Entry one   & Entry two   \\
%    Entry three & Entry four  \\
%    Entry five  & Entry five  \\
%    Entry seven & Entry eight \\
%    \hline
%  \end{tabular}
%\end{table}
%
%Adding notes to tables can be complicated.  Perhaps the easiest
%method is to generate these using the basic
%\texttt{\textbackslash textsuperscript} and
%\texttt{\textbackslash emph} macros, as illustrated (Table~\ref{tbl:notes}).
%\begin{table}
%  \caption{A table with notes}
%  \label{tbl:notes}
%  \begin{tabular}{ll}
%    \hline
%    Header one                            & Header two \\
%    \hline
%    Entry one\textsuperscript{\emph{a}}   & Entry two  \\
%    Entry three\textsuperscript{\emph{b}} & Entry four \\
%    \hline
%  \end{tabular}
%
%  \textsuperscript{\emph{a}} Some text;
%  \textsuperscript{\emph{b}} Some more text.
%\end{table}
%
%The example file also loads the optional \textsf{mhchem} package, so
%that formulas are easy to input: \texttt{\textbackslash ch\{H2SO4\}}
%gives \ch{H2SO4}.  See the use in the bibliography file (when using
%titles in the references section).
%
%The use of new commands should be limited to simple things which will
%not interfere with the production process.  For example,
%\texttt{\textbackslash mycommand} has been defined in this example,
%to give italic, mono-spaced text: \mycommand{some text}.
%
%\section{Extra information when writing JACS Communications}
%
%When producing communications for \emph{J.~Am.\ Chem.\ Soc.}, the
%class will automatically lay the text out in the style of the
%journal. This gives a guide to the length of text that can be
%accommodated in such a publication. There are some points to bear in
%mind when preparing a JACS Communication in this way.  The layout
%produced here is a \emph{model} for the published result, and the
%outcome should be taken as a \emph{guide} to the final length. The
%spacing and sizing of graphical content is an area where there is
%some flexibility in the process.  You should not worry about the
%space before and after graphics, which is set to give a guide to the
%published size. This is very dependant on the final published layout.
%
%You should be able to use the same source to produce a JACS
%Communication and a normal article.  For example, this demonstration
%file will work with both \texttt{type=article} and
%\texttt{type=communication}. Sections and any abstract are
%automatically ignored, although you will get warnings to this effect.
%
%%%%%%%%%%%%%%%%%%%%%%%%%%%%%%%%%%%%%%%%%%%%%%%%%%%%%%%%%%%%%%%%%%%%%%
%%% The "Acknowledgement" section can be given in all manuscript
%%% classes.  This should be given within the "acknowledgement"
%%% environment, which will make the correct section or running title.
%%%%%%%%%%%%%%%%%%%%%%%%%%%%%%%%%%%%%%%%%%%%%%%%%%%%%%%%%%%%%%%%%%%%%%
\begin{acknowledgement}
The authors thank Professor Lindsey Madison of Colby College for her time and expertise.
\end{acknowledgement}
%
%%%%%%%%%%%%%%%%%%%%%%%%%%%%%%%%%%%%%%%%%%%%%%%%%%%%%%%%%%%%%%%%%%%%%%
%%% The same is true for Supporting Information, which should use the
%%% suppinfo environment.
%%%%%%%%%%%%%%%%%%%%%%%%%%%%%%%%%%%%%%%%%%%%%%%%%%%%%%%%%%%%%%%%%%%%%%
%\begin{suppinfo}
%
%A listing of the contents of each file supplied as Supporting Information
%should be included. For instructions on what should be included in the
%Supporting Information as well as how to prepare this material for
%publications, refer to the journal's Instructions for Authors.
%
%The following files are available free of charge.
%\begin{itemize}
%  \item Filename: brief description
%  \item Filename: brief description
%\end{itemize}
%
%\end{suppinfo}
%
%%%%%%%%%%%%%%%%%%%%%%%%%%%%%%%%%%%%%%%%%%%%%%%%%%%%%%%%%%%%%%%%%%%%%%
%%% The appropriate \bibliography command should be placed here.
%%% Notice that the class file automatically sets \bibliographystyle
%%% and also names the section correctly.
%%%%%%%%%%%%%%%%%%%%%%%%%%%%%%%%%%%%%%%%%%%%%%%%%%%%%%%%%%%%%%%%%%%%%%
\bibliography{rs_bib}

\end{document}
